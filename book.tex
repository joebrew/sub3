\documentclass[]{tufte-handout}

% ams
\usepackage{amssymb,amsmath}

\usepackage{ifxetex,ifluatex}
\usepackage{fixltx2e} % provides \textsubscript
\ifnum 0\ifxetex 1\fi\ifluatex 1\fi=0 % if pdftex
  \usepackage[T1]{fontenc}
  \usepackage[utf8]{inputenc}
\else % if luatex or xelatex
  \makeatletter
  \@ifpackageloaded{fontspec}{}{\usepackage{fontspec}}
  \makeatother
  \defaultfontfeatures{Ligatures=TeX,Scale=MatchLowercase}
  \makeatletter
  \@ifpackageloaded{soul}{
     \renewcommand\allcapsspacing[1]{{\addfontfeature{LetterSpace=15}#1}}
     \renewcommand\smallcapsspacing[1]{{\addfontfeature{LetterSpace=10}#1}}
   }{}
  \makeatother
\fi

% graphix
\usepackage{graphicx}
\setkeys{Gin}{width=\linewidth,totalheight=\textheight,keepaspectratio}

% booktabs
\usepackage{booktabs}

% url
\usepackage{url}

% hyperref
\usepackage{hyperref}

% units.
\usepackage{units}


\setcounter{secnumdepth}{-1}

% citations
\usepackage{natbib}
\bibliographystyle{plainnat}

% pandoc syntax highlighting
\usepackage{color}
\usepackage{fancyvrb}
\newcommand{\VerbBar}{|}
\newcommand{\VERB}{\Verb[commandchars=\\\{\}]}
\DefineVerbatimEnvironment{Highlighting}{Verbatim}{commandchars=\\\{\}}
% Add ',fontsize=\small' for more characters per line
\newenvironment{Shaded}{}{}
\newcommand{\KeywordTok}[1]{\textcolor[rgb]{0.00,0.44,0.13}{\textbf{{#1}}}}
\newcommand{\DataTypeTok}[1]{\textcolor[rgb]{0.56,0.13,0.00}{{#1}}}
\newcommand{\DecValTok}[1]{\textcolor[rgb]{0.25,0.63,0.44}{{#1}}}
\newcommand{\BaseNTok}[1]{\textcolor[rgb]{0.25,0.63,0.44}{{#1}}}
\newcommand{\FloatTok}[1]{\textcolor[rgb]{0.25,0.63,0.44}{{#1}}}
\newcommand{\ConstantTok}[1]{\textcolor[rgb]{0.53,0.00,0.00}{{#1}}}
\newcommand{\CharTok}[1]{\textcolor[rgb]{0.25,0.44,0.63}{{#1}}}
\newcommand{\SpecialCharTok}[1]{\textcolor[rgb]{0.25,0.44,0.63}{{#1}}}
\newcommand{\StringTok}[1]{\textcolor[rgb]{0.25,0.44,0.63}{{#1}}}
\newcommand{\VerbatimStringTok}[1]{\textcolor[rgb]{0.25,0.44,0.63}{{#1}}}
\newcommand{\SpecialStringTok}[1]{\textcolor[rgb]{0.73,0.40,0.53}{{#1}}}
\newcommand{\ImportTok}[1]{{#1}}
\newcommand{\CommentTok}[1]{\textcolor[rgb]{0.38,0.63,0.69}{\textit{{#1}}}}
\newcommand{\DocumentationTok}[1]{\textcolor[rgb]{0.73,0.13,0.13}{\textit{{#1}}}}
\newcommand{\AnnotationTok}[1]{\textcolor[rgb]{0.38,0.63,0.69}{\textbf{\textit{{#1}}}}}
\newcommand{\CommentVarTok}[1]{\textcolor[rgb]{0.38,0.63,0.69}{\textbf{\textit{{#1}}}}}
\newcommand{\OtherTok}[1]{\textcolor[rgb]{0.00,0.44,0.13}{{#1}}}
\newcommand{\FunctionTok}[1]{\textcolor[rgb]{0.02,0.16,0.49}{{#1}}}
\newcommand{\VariableTok}[1]{\textcolor[rgb]{0.10,0.09,0.49}{{#1}}}
\newcommand{\ControlFlowTok}[1]{\textcolor[rgb]{0.00,0.44,0.13}{\textbf{{#1}}}}
\newcommand{\OperatorTok}[1]{\textcolor[rgb]{0.40,0.40,0.40}{{#1}}}
\newcommand{\BuiltInTok}[1]{{#1}}
\newcommand{\ExtensionTok}[1]{{#1}}
\newcommand{\PreprocessorTok}[1]{\textcolor[rgb]{0.74,0.48,0.00}{{#1}}}
\newcommand{\AttributeTok}[1]{\textcolor[rgb]{0.49,0.56,0.16}{{#1}}}
\newcommand{\RegionMarkerTok}[1]{{#1}}
\newcommand{\InformationTok}[1]{\textcolor[rgb]{0.38,0.63,0.69}{\textbf{\textit{{#1}}}}}
\newcommand{\WarningTok}[1]{\textcolor[rgb]{0.38,0.63,0.69}{\textbf{\textit{{#1}}}}}
\newcommand{\AlertTok}[1]{\textcolor[rgb]{1.00,0.00,0.00}{\textbf{{#1}}}}
\newcommand{\ErrorTok}[1]{\textcolor[rgb]{1.00,0.00,0.00}{\textbf{{#1}}}}
\newcommand{\NormalTok}[1]{{#1}}

% longtable

% multiplecol
\usepackage{multicol}

% strikeout
\usepackage[normalem]{ulem}

% morefloats
\usepackage{morefloats}


% tightlist macro required by pandoc >= 1.14
\providecommand{\tightlist}{%
  \setlength{\itemsep}{0pt}\setlength{\parskip}{0pt}}

% title / author / date
\title{Distance running: a new approach to training}
\author{Joe Brew}
\date{2017-03-10}


\begin{document}

\maketitle




\begin{Shaded}
\begin{Highlighting}[]
\NormalTok{knitr::opts_chunk$}\KeywordTok{set}\NormalTok{(}\DataTypeTok{comment =} \OtherTok{NA}\NormalTok{, }\DataTypeTok{echo =} \OtherTok{FALSE}\NormalTok{, }
    \DataTypeTok{warning =} \OtherTok{FALSE}\NormalTok{, }\DataTypeTok{message =} \OtherTok{FALSE}\NormalTok{, }\DataTypeTok{error =} \OtherTok{TRUE}\NormalTok{, }
    \DataTypeTok{cache =} \OtherTok{FALSE}\NormalTok{)}
\end{Highlighting}
\end{Shaded}

\section{Introduction}\label{introduction}

\subsection{The importance of injury}\label{the-importance-of-injury}

Almost all runners experience injury at some point. For \emph{most}
runners, injury is the primary limiting factor on performance. In high
school and university teams, it is not uncommon to have up to half a
team injured at any given time. Injury is such an ever-present factor of
distance running that many runners take up quasi-superstitious routines
to ensure its prevention, such as over-protocolozing warm-up and
cooldown routines, stretching, etc.

\subsection{Elite runners are
different}\label{elite-runners-are-different}

Elite\footnote{``Elite'' means able to run 1500 meters in 3:40 (men) /
  4:00 (women), or the marathon in 2:15 (men) / 2:35 (women).} runners
are exceptional. While much of what sets them apart from ``normal''
runners has to do with the physiological components which lend
themselves to speed and endurance (muscle fibers, heart, lungs, etc.),
an often overlooked characteristic that most elite runners possess (and
most normal runners do not) is an \emph{abnormally low risk of injury}.
Many runners are naturally good; the great ones become so because they
are able to tolerate hard training without succombing to injury.

The limiting factor on many elites' performance is fatigue induced by
overtraining. For normal runners, injury occurs long before overtraining
fatigue is ever reached.

\subsection{Elite runners should not be
copied}\label{elite-runners-should-not-be-copied}

Most training plans for normal runners are essentially copies of elite
training plans, but with a slight reduction in volume and intensity. If
elites build a ``base'' during the off-season with long, slow miles,
then so should normal runners, right? And if elites run 20 kilometers on
an ``easy'' day between workouts, then surely 10 kilometers is
appropriate for a non-elite?

The if-the-elites-are-doing-it-so-should-I attitude has caused far more
harm than good among distance runners. High school junior varsity
runners often don't make it to their first cross country meet because
they are nursing an injury from running too much volumne during the
``base''. Talented college athletes often make it part way through a
season of hard training before having to taper off due to the appearance
of multiple semi-injuries. Adult runners often cycle in and out of
phases of improvement, injury, and regression.

Normal runners are different from elite runners, and their training
plans should be different too. Not just in volume and intensity, but
\emph{in concept}. Whereas an elite training plans aims to maximize
speed and endurance, a non-elite training plan should aim, first and
foremost, to minimize the risk of injury.

\subsection{More bang for your buck}\label{more-bang-for-your-buck}

In this book, I outline the three core dimensions of training for
distance running:

\begin{enumerate}
\def\labelenumi{\arabic{enumi}.}
\tightlist
\item
  Volume
\item
  Intensity
\item
  Concentration
\end{enumerate}

Each of these dimensions is closely related to an athlete's fitness as
well as her risk of injury, but not in the same ways. By identifying and
quantifying the relationship between each dimension and its effect on
fitness (``benefit'') and on risk of injury (``cost''), we devise a new
framework for training distance runners, radically different from the
orthodoxy.

Though it is not expressly written for working professionals, busy
stuents or stressed out parents, and does not explicitly stive to reduce
the amount of time dedicated to training, a pleasant side effect of its
focus on low volume, high-intensity and concentraing workouts into fewer
sessions is that it fits well into the schedules of those for whom
finding the time to train is a chief obstacle in reaching their goals.

This framework, whose philosophy can be summed up simply as ``getting
the most bang for your buck'', is laid out in the following pages. Enjoy

\section{The 3 dimensions of
training}\label{the-3-dimensions-of-training}

The effect that a training plan has on a runner is a function of the
following 3 dimensions:

\begin{enumerate}
\def\labelenumi{\arabic{enumi}.}
\tightlist
\item
  \emph{Volume:} The total amount of running (measured in minutes,
  miles, kilometers, etc.)
\item
  \emph{Intensity:} The speed of running (measured in pace)
\item
  \emph{Concentration:} The ratio of total volume and intensity to
  number of sessions
\end{enumerate}

An increase in any dimension makes a training plan ``harder''. And, in
general, a ``harder'' plan leads to athletes who are both fitter and
more injury prone.

Traditionally, in regards to our 3 dimensions, most amateur training
plans are simple copies of elite programs.

\includegraphics{book_files/figure-latex/unnamed-chunk-2-1}

For example, whereas an elite athlete marathoner might run 150
kilometers per week, a high school runner might cover 75 (volume). The
elite may run perhaps 20\% of their overall volumne at 5k pace or
faster, and the high school runner will do identically. The elite will
likely train 6 days a week, one session per day; and the high school
runner - the exact same thing.

Most taditional training plans strive to identify the ``sweet spot'' -
somewhere between nothing and 24 hours a day of intense training - where
the risk of injury is acceptably low yet the training is sufficiently
hard so as to induce adaptive response.

\includegraphics{book_files/figure-latex/unnamed-chunk-3-1}

In other words, a traditional training plan simply plots a runner along
the red line. Runners whose tolerance for injury is greatest get
assigned to more difficult programs, while runners who are more injury
prone end up more to the left on the chart (ie, less difficult
programs). Elites, with their inherently high resistance to injury, get
assigned to the hardest programs.

What traditional programs fail to do is realize that \emph{the shape of
the curve can be changed}. By manipulating the 3 dimensions of training
- volume, intensity, and concentration - one can achieve a ``harder''
(and therefore more effective) training regimen without necessarily
increasing risk of injury.

\subsection{The effect of the
dimensions}\label{the-effect-of-the-dimensions}

In order to appropriately combine volume, intensity and concentration,
we must first understand how each dimension affects the runner, both in
terms of increasing fitness as well as increasing injury.

\subsubsection{Volume}\label{volume}

The more you run, the better you get at it. This simple truth is
apparent to everyone, from the first-time 5k runner to the elite
marathoner. However, though the effect of volume on fitness can be
tremendous, so too can its effect on the likelihood of injury. Too much
volume is likely the main culprit of most running injuries.

The below chart shows the cost-benefit relationship and its association
with volume. The more one runs, the more benefit (blue); however, just
as volume increases fitness, so to does it increase the likelihood
injury. The relationship between volumne and cost/benefit is linear:
doubling volume ``doubles'' ones fitness, but also his risk of injury.

\includegraphics{book_files/figure-latex/unnamed-chunk-4-1}

\subsection{Intensity}\label{intensity}

The relationship of intensity\footnote{(speed)} with fitness and
injury\footnote{(cost and benefit)} is slightly more complex. If one
keeps the same volume, but increases the amount of time spent running at
race pace, both her fitness and likelihood of injury will increase.
Unlike volume, however, fitness and injury likelihood do not travel
collinearly; rather, the gains associated with increased intensity are
greater than the augmented risk of injury. In other words, with
increased intensity, the risk of injury grows linearly, but the
physiological benefits grow exponentially.

\includegraphics{book_files/figure-latex/unnamed-chunk-5-1}

\subsection{Concentration}\label{concentration}

Concentration is one of the most overlooked dimensions of distance
running. Nearly all runners know their total volume (``weekly mileage'')
and intensity (amount spent running at different paces), but very few
are fully cognizant of the importance of session concentration. This is
a surprise, since intuitively all runners know that running a 20k is
harder than running 4 5ks.

By copying elites, most runners are spreading out their workouts into
too many sessions. By concentraing the same volume and intensity into
fewer sessions, however, a runner can increase fitness substantially
without significantly increasing the risk of injury. In other words, by
reducing the number of workout days in a week, a runner kills two birds
with one stone: she both (a) increases the amount of uninterrupted rest
time her body has to recover between sessions, and (b) she increases the
difficulty of each session.

\includegraphics{book_files/figure-latex/unnamed-chunk-6-1}

\section{From traditional to
transformational}\label{from-traditional-to-transformational}

Let's work through an example. Jack is a 10k runner who trains 4 days a
week. Each of his four runs are identical in length, and he spends about
a quarter of each run going at race pace.

\includegraphics{book_files/figure-latex/unnamed-chunk-7-1}

Jack wants to improve. A traditional program might suggest that Jack
should increase his volume. If Jack doubles his mileage, without
changing any other aspect of his training program, he'll get
significantly faster.

\begin{marginfigure}
\includegraphics{book_files/figure-latex/unnamed-chunk-8-1} \end{marginfigure}

However, by going from baseline to LSD (long slow distance), Jack also
doubled his risk of injury (since injury and fitness are collinear when
it comes to volumne).

\includegraphics{book_files/figure-latex/unnamed-chunk-9-1}

To reduce the risk of injury, Jack explores an alternative option:
keeping his mileage constant, but increasing his speed. Now, instead of
running only the last quarter of each run at race pace, he runs half of
each run at race pace.

\begin{marginfigure}
\includegraphics{book_files/figure-latex/unnamed-chunk-10-1} \end{marginfigure}

The improvement in fitness, relative to the LSD plan, is similar.
However, the risk of injury is much lower than in the LSD plan.

\includegraphics{book_files/figure-latex/unnamed-chunk-11-1}

Having settled firmly on the side of intensity in the volume
vs.~intensity debate, Jack wonders how else he can improve his fitness.
One way is to take concentrate his workouts - the same volume and
intensity - into fewer sessions.

\begin{marginfigure}
\includegraphics{book_files/figure-latex/unnamed-chunk-12-1} \end{marginfigure}

By eliminating one run (ie, keeping the total weekly volume the same by
increasing slightly the amount run in each session), Jack manages to
increase recovery time and workout difficuty. The ``cost'' of this
approach is low (remember, his overall volume is identical to the
``baseline'' example), but the advantages are great.

\includegraphics{book_files/figure-latex/unnamed-chunk-13-1}

How else can Jack tweak his program? He has already touched on all three
dimensions, having decided to keep volume constant, increase intensity,
and concentrate his workouts into fewer sessions\ldots{}

Jack can improve his fitness even more, with only minor increased risk
of injury, by \emph{combining} the dimensions of intensity, volume and
concentration. The overall volume, and intensity remain the same as the
previous example, but Jack varies his workouts by increasing and
decreasing their overall times, as well as varying the amount of time in
each workout spent at race pace.

\begin{verbatim}
# A tibble: 1 × 2
   cost  benefit
  <dbl>    <dbl>
1   150 341.7114
\end{verbatim}

\begin{marginfigure}
\includegraphics{book_files/figure-latex/unnamed-chunk-14-1} \end{marginfigure}

\includegraphics{book_files/figure-latex/unnamed-chunk-15-1}

\section{Compare Jack's original plan (baseline) with his final
(optimal)
one.}\label{compare-jacks-original-plan-baseline-with-his-final-optimal-one.}

\includegraphics{book_files/figure-latex/unnamed-chunk-16-1}

The overall weekly volume (amount of time spent running) is identical,
the intensity is slightly greater, and the concentration is greater. The
``cost'' of the ``optimal'' workout plan is only slightly higher than in
the ``baseline'' plan, but the benefits are far greater.

\bibliography{skeleton.bib}



\end{document}
